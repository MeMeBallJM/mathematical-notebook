\documentclass{article}
\usepackage[a4paper,margin=1in,footskip=0.25in]{geometry}

\input{latex-template/index.tex}

\usepackage{enumerate}

\newcommand{\tp}{\mathcal{T}}
\newcommand{\pset}[1]{\mathcal{P}(#1)}
\newcommand{\defeq}{\stackrel{\text{def}}{=}}

\begin{document}

\begin{definition}{Topological Space}{}
  A \emph{topology} $\tp \subseteq \pset{X}$ on a set $X$ forms a \emph{topological space} $(X, \tp)$ if it satisfies the following,

  \begin{enumerate}[(i)]
    \item $\emptyset, X \in \tp$.
    \item If $S \subseteq \tp$ then $\bigcup S \in \tp$.
    \item If $S_1, S_2 \in \tp$ then $S_1 \cap S_2 \in \tp$.
  \end{enumerate}
\end{definition}

\begin{definition}{Basis for a Topology}{}
  A subset $\beta \subseteq \tp$ for a topology $(X, \tp)$ is a basis of the topology if every open set can be represented as a union of the elements of $\beta$.
  That is, for all open sets $\mathcal{O} \in \tp$ there exists a subset $E \subseteq \beta$ such that $\bigcup E = \mathcal{O}$.
  Elements of $\beta$ are called \emph{basic open sets}.
\end{definition}


\end{document}